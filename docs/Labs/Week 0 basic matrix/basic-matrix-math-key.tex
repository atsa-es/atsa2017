\input{../tex/headerfile}
\input{../tex/mathdefs}
\setcounter{MaxMatrixCols}{20}
\usepackage{enumerate}
\usepackage{Sweave}
\begin{document}
\Sconcordance{concordance:basic-matrix-math-key.tex:basic-matrix-math-key.Rnw:%
1 4 1 1 0 2 1}
\Sconcordance{concordance:basic-matrix-math-key.tex:./basic_matrix_solns-key.xRnw:ofs 8:%
1 1 10 9 1 1 2 4 0 2 2 4 0 2 2 4 0 1 2 1 1 1 3 2 0 1 2 4 0 1 2 1 1 1 2 1 0 1 1 3 0 1 2 1 1 1 2 4 0 1 2 1 1 %
1 2 4 0 1 2 1 1 1 2 4 0 1 2 1 1 1 2 4 0 1 2 1 1 1 2 1 0 1 2 1 0 2 1 3 0 1 2 1 1 1 2 1 0 1 2 4 0 1 2 1 1 1 2 %
1 0 1 1 3 0 1 2 1 1 1 2 4 0 1 2 1 1 1 2 1 0 2 1 1 2 4 0 1 2 1 1 1 3 5 0 1 2 1 1 1 3 2 0 6 1 3 0 1 2 1 1 1 3 %
2 0 2 1 3 0 1 2 1 1 1 4 3 0 2 1 3 0 1 2 1 1 1 4 3 0 2 1 3 0 1 2 1 1 1 2 1 0 1 1 1 3 2 0 1 1 3 0 1 2 3 1}
\Sconcordance{concordance:basic-matrix-math-key.tex:basic-matrix-math-key.Rnw:ofs 198:%
9 3 1}



\chapter*{Solutions Chapter \ref{chap:basicmat}}
\addcontentsline{toc}{chapter}{Solutions Chapter \ref{chap:basicmat}}

%######################################
%# Basic Matrix Math
%# Homework Questions
%######################################

\begin{enumerate}
\item 
\begin{Schunk}
\begin{Sinput}
 A=matrix(1:4,4,3)
\end{Sinput}
\end{Schunk}
\item 
\begin{Schunk}
\begin{Sinput}
 A[1:2,1:2]
\end{Sinput}
\end{Schunk}
\item
\begin{Schunk}
\begin{Sinput}
 A=matrix(1:12,4,3, byrow=TRUE)
\end{Sinput}
\end{Schunk}

\item
\begin{Schunk}
\begin{Sinput}
 #end up with a vector
 A[3,]
 #end up with a matrix
 A[3,,drop=FALSE]
\end{Sinput}
\end{Schunk}

\item
\begin{Schunk}
\begin{Sinput}
 B=matrix(1,4,3)
 B[2,3]=2
\end{Sinput}
\end{Schunk}

\item 
\begin{Schunk}
\begin{Sinput}
 t(B)
\end{Sinput}
\end{Schunk}

\item
\begin{Schunk}
\begin{Sinput}
 diag(1:4)
\end{Sinput}
\end{Schunk}

\item
\begin{Schunk}
\begin{Sinput}
 B=diag(1,5)
\end{Sinput}
\end{Schunk}

\item
\begin{Schunk}
\begin{Sinput}
 diag(B)=2
\end{Sinput}
\end{Schunk}

\item
\begin{Schunk}
\begin{Sinput}
 diag(1,4)+1
 #or
 B=matrix(1,4,4)
 diag(B)=2
 B
\end{Sinput}
\end{Schunk}

\item
\begin{Schunk}
\begin{Sinput}
 solve(B)
 #or this but only works because B is symmetric
 chol2inv(chol(B))
\end{Sinput}
\end{Schunk}

\item
\begin{Schunk}
\begin{Sinput}
 B=matrix(letters[1:9],3,3)
 B
\end{Sinput}
\end{Schunk}

\item
\begin{Schunk}
\begin{Sinput}
 diag(B)="cat"
\end{Sinput}
\end{Schunk}

\item
\begin{Schunk}
\begin{Sinput}
 A=matrix(1,4,3)
 B=matrix(2,3,4)
 A%*%B
 #or
 B%*%A
\end{Sinput}
\end{Schunk}

\item
\begin{Schunk}
\begin{Sinput}
 # A%*%A #throws an error
 A%*%t(A) #works
\end{Sinput}
\end{Schunk}

\item
\begin{Schunk}
\begin{Sinput}
 #this is an example where you use B to select values in A
 A=matrix(1:9,3,3)
 B=matrix(0,3,3)
 B[1,1]=1
 B[2,3]=1
 B[3,2]=1
 C=A%*%B
 diag(C)
\end{Sinput}
\end{Schunk}

\item
\begin{Schunk}
\begin{Sinput}
 #this shows one of the uses of diagonal matrices
 B=diag(2,3)
 C=A%*%B
 C
\end{Sinput}
\end{Schunk}

\item
\begin{Schunk}
\begin{Sinput}
 #this shows how to use a column vector (matrix with 1 col) 
 #to compute row sums
 B=matrix(1,3,1)
 C=A%*%B
 C
\end{Sinput}
\end{Schunk}

\item
\begin{Schunk}
\begin{Sinput}
 #this shows how to use a row vector (matrix with one row) 
 #to compute column sums
 B=matrix(1,1,3)
 C=B%*%A
 C
\end{Sinput}
\end{Schunk}
 
\item
\begin{Schunk}
\begin{Sinput}
 A=diag(1,3)+1
 C=matrix(3,3,1)
 #AB=C
 #B=inv(A)%*%C
 B=solve(A)%*%C
 B
\end{Sinput}
\end{Schunk}

\end{enumerate}



\bibliography{../tex/Fish507}

\end{document}
