\input{../tex/headerfile}
\input{../tex/mathdefs}
\usepackage{Sweave}
\begin{document}
\Sconcordance{concordance:template_lab_DLM.tex:template_lab_DLM.Rnw:%
1 2 1 1 0 2 1}
\Sconcordance{concordance:template_lab_DLM.tex:./Lab_6_intro_DLMs.Rnw:ofs 6:%
1 1 7 1 2 60 1 1 8 11 0 1 2 25 1 1 2}
\Sconcordance{concordance:template_lab_DLM.tex:template_lab_DLM.Rnw:ofs 108:%
7 3 1}



\input{./figures/MLR--RUNFIRST}

\chapter{Introduction to Dynamic Linear Models (DLMs)}



%==========================
\section{Homework problems}
%==========================

For the homework this week we will use a DLM to examine some of the time-varying properties of the spawner-recruit relationship for Pacific salmon.  Much work has been done on this topic, particularly by Randall Peterman and his students and post-docs at Simon Fraser University.  To do so, researchers commonly use a Ricker model because of its relatively simple form, such that the number of recruits (offspring) born in year $t$ ($R_t$) from the number of spawners (parents) ($S_t$) is

\begin{equation}\label{eqn:baseRicker}
R_t = a S_t e^{-b S + v_t}.
\end{equation}


\noindent The parameter $a$ determines the maximum reproductive rate in the absence of any density-dependent effects (the slope of the curve at the origin), $b$ is the strength of density dependence, and $v_t \sim N(0,\sigma)$.  In practice, the model is typically log-transformed so as to make it linear with respect to the predictor variable $S_t$, such that

\begin{equation}\label{eqn:lnRicker}
\begin{aligned}
\text{log}(R_t) &= \text{log}(a) + \text{log}(S_t) -b S_t + v_t \\
\text{log}(R_t) - \text{log}(S_t) &= \text{log}(a) -b S_t + v_t \\
\text{log}(R_t/S_t) &= \text{log}(a) - b S_t + v_t.
\end{aligned}
\end{equation}


\noindent Substituting $y_t = \text{log}(R_t/S_t)$, $x_t = S_t$, and $\alpha = \text{log}(a)$ yields a simple linear regression model with intercept $\alpha$ and slope $b$.

Unfortunately, however, residuals from this simple model typically show high-autocorrelation due to common environmental conditions that affect overlaping generations.  Therefore, to correct for this and allow for an index of stock productivity that controls for any density-dependent effects, the model may be re-witten as

\begin{equation}\label{eqn:lnTVRicker}
\begin{aligned}
\text{log}(R_t/S_t) &= \alpha_t - b S_t + v_t, \\
\alpha_t &= \alpha_{t-1} + w_t,
\end{aligned}
\end{equation}

\noindent and $w_t \sim N(0,q)$.  By treating the brood-year specific productivity as a random walk, we allow it to vary, but in an autocorrelated manner so that consecutive years are not independent from one another.

More recently, interest has grown in using covariates ($e.g.$, sea-surface temperature) to explain the interannual variability in productivity.  In that case, we can can write the model as

\begin{equation}\label{eqn:lnCovRicker}
\text{log}(R_t/S_t) = \alpha + \delta_t X_t - b S_t + v_t.
\end{equation}

\noindent In this case we are estimating some base-level productivity ($\alpha$) plus the time-varying effect of some covariate $X_t$ ($\delta_t$). 


%--------------------------------
\subsection{Spawner-recruit data}
%--------------------------------

The data come from a large public database begun by Ransom Myers many years ago.  If you are interested, you can find lots of time series of spawning-stock, recruitment, and harvest for a variety of fishes around the globe.  Here is the website:\\

\verb@http://ram.biology.dal.ca/~myers/about_site.html@\\

For this exercise, we will use spawner-recruit data for sockeye salmon ($Oncorhynchus nerka$) from the Fraser River in British Columbia. Specifically, the data come from a population in the Chilko River and span the years 1948-1986.  In addition, we'll examine the potential effects of the Pacific Decadal Oscillation (PDO) during the salmon's first year in the ocean, which is widely believed to be a "bottleneck" to survival.

Here are the data:

\begin{Schunk}
\begin{Sinput}
 # get S-R data; cols are:
 # 1: brood yr (brood.yr)
 # 2: number of spawners (Sp)
 # 3: number of recruits (Rec)
 # 4: PDO during first summer at sea (PDO.t2)
 # 5: PDO during first winter at sea (PDO.t3)
 load("ChilkoSockeye.RData")
\end{Sinput}
\end{Schunk}



%---------------------
\subsection{Questions}
%---------------------

\noindent Use the information above to do the following:

\begin{enumerate} \itemsep5pt \parskip0pt \parsep0pt

\item Begin by fitting a reduced form of Equation \ref{eqn:lnTVRicker} that includes only a time-varying level ($\alpha_t$) and observation error ($v_t$).  Although you will be modeling productivity as completely independent of their parents, it will provde some insights as to the overall temporal pattern in recruitment.  Plot the ts of $\alpha_t$ and note the AICc for this model.

\item Fit the full model specified by Equation \ref{eqn:lnTVRicker}.  For this model, obtain the time series of $\alpha_t$, which is an estimate of the stock productivity in the absence of density-dependent effects. How do these estimates of productivity compare to those from the previous question?  Plot the ts of $\alpha_t$ and note the AICc for this model.  ($Hint$: If you don't want a parameter to vary with time, what does that say about its process variance?.)

\item Fit the model specified by Equation \ref{eqn:lnCovRicker} with the summer PDO index as the covariate (\texttt{PDO.t2}). What is the mean level of productivity?  Plot the ts of $\delta_t$ and note the AICc for this model

\item Fit the model specified by Equation \ref{eqn:lnCovRicker} with the winter PDO index as the covariate (\texttt{PDO.t3}). What is the mean level of productivity?  Plot the ts of $\delta_t$ and note the AICc for this model


\end{enumerate}


%----------------
% end of content
%----------------

\input{./figures/MLR--reset}

\bibliography{../tex/Fish507}

\end{document}
